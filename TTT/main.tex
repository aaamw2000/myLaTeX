\documentclass[12pt]{article}
\usepackage{graphicx}
\usepackage[margin=2cm]{geometry}
\usepackage{titlesec}
\usepackage{titling}
\usepackage{lipsum}
\usepackage{fancyhdr}
\usepackage{rotating}
\usepackage{hyperref}
\usepackage{enumitem}
\usepackage{tikz}
\usepackage{xcolor}
% \usepackage{standalone}
\usepackage{pdfpages}

\graphicspath{{./resources/}}

\newcommand{\bb}[1]{\textbf{\textit{#1}}}
\newcommand{\telet}[0]{Teletarget }
\newcommand{\tm}[0]{telemarketing }
\newcommand{\you}[1]{{\noindent\color{blue} #1}}
\newcommand{\owner}[1]{{\noindent\ #1}}
\newcommand{\gowner}[1]{{\noindent\color{green} #1}}
\newcommand{\bowner}[1]{{\noindent\color{red} #1}}
\newcommand{\jargon}[1]{{\large\noindent\color{red} #1:}}

\hypersetup{
    colorlinks=true,
    linkcolor=blue,
    filecolor=magenta,      
    urlcolor=cyan,
    pdftitle={miniMBAProject},
    pdfpagemode=FullScreen,
    }

\lhead{Teletarget}
\chead{}
\rhead{\includegraphics[scale=0.02]{tt_coldcalling.jpg}}
\lfoot{}
\cfoot{\thepage}
\rfoot{}

\renewcommand{\thefigure}{}

\renewcommand{\maketitle}{
\begin{center}
    \Huge\textbf{Training material for newcomers} \\ 

    \vspace{1em}
\end{center}
}


\title{MBA Assignment}
\author{Amr Wahdan}
\date{Thursday, May 11th 2023}

\begin{document}
\pagenumbering{0}

\begin{center}
   \topskip12cm
    \includegraphics[width=0.6\textwidth]{tt_coldcalling.jpg}  
\end{center}
\vspace{1em}
\begin{center}
    \scalebox{2}{\Huge\textbf{Teletarget}}
\end{center}
\vspace{1em}
\maketitle
\pagenumbering{roman}
\newpage

\begin{center}
   \topskip8cm
    \includegraphics[width=0.7\textwidth]{headset.png}  
\end{center}
\vspace{1em}

\begin{center}
    \Large\bfseries History of Telemarketing
    \vspace{0.5em}
    \hrule\vspace{0.2em}\hrule
\end{center}

\begin{itemize}
    \item Telemarketing first arose with the popularization and recent upgrade of Telephones and the invention of the internet in the 70's. 

    \item It first started with marketing people utilizing phone books, and cold calling people to find if they're interested in buying the marketer's product.

    \item Back then, \tm\ used to be of little to no use when compared with the amount of effort put in by the telemarketer. Especially since the calls were mostly random, and there was no way for the telemarketer to target his calls to potential leads AKA buyers/sellers. 

    \item However, with the internet gaining steam, and being more and more popular everyday, data brokers arose. These data brokers collect data from big tech companies in various fields, and compile lists of people sharing a specific criteria.

    \item And with the rise of data brokers, and targeted ads, \tm\ started to make much more sense. Now, you can buy a list of people who own pets for example, and call them to see if they would like a free vet check-up at your clinic. Or an example more close to home would be that you can now call people who listed their properties previously, and ask them if they would like to sell it or not. Since chances are, not everyone who listed their property got to sell it. And some would be more than happy to discuss it with you. And you can find really good deals sometimes.
    
\end{itemize}


\newpage
\pagenumbering{arabic}
\pagestyle{fancy}
\topskip2cm

\section{The pillars of a qualified lead}
A lead is basically self explanatory --- it is something that \textbf{LEADS}. So, a lead is basically a set of information about potential customers that inform you of how likely they might become actual customers. And in our industry --- the real estate industry --- a lead is the collection of facts you need to know about a customer to infer that he's likely to sell his property.

\begin{center}
    \includegraphics[width=0.45\textwidth]{dollars.png}
\end{center}

\subsection{Qualities of a qualified lead}

    The most important thing about a qualified lead is \bb{INTEREST}. Without interest, a lead is certainly not qualified. You have to sense interest, and actual genuine motivation to sell, in the person you're talking to. That could be them giving you lots of information without you asking for it, or them talking to you about how much they want to sell, and tried multiple times to sell to no avail. It could also be just cooperating with you on the call, and answering your questions with no disruptions.\ \par

    The second most important thing to qualify a lead is the owner asking for or talking about \bb{REASONABLE} things. That could be them wanting a reasonable cash offer for their property, or them telling you information that make sense about the property that loosely conform with the info you have available online.\ \par
\vspace{2em}
\noindent
    \bb{INTEREST} \& \bb{REASONABLENESS} are the two most fundamental things about a lead. Other than these two, there are some stuff you should check for. But without these two, a lead is certainly disqualified.

\newpage
    {\Large\textbf{What to check for:}}
    \vspace{1em}
\begin{enumerate}[label= \roman*.]
    \item \bb{Asking Price (AP)}:
    The asking price is the dollar amount the owner demands to sell his property. That price changes depending on the type of the offer. So, if you're going to buy a property on installments, you're going to have to pay more, hence a bigger AP\@. However, if you're going to pay fully in cash, that usually corresponds with a smaller AP\@. And we assess the AP by comparing it with the \bb{Market Value (MV)}. To find the MV for a property, one has to get the property appraised by a licensed real-estate agent. And that appraisal is not free. We, however, find the market value by looking up the address of the property on \href{zillow.com}{Zillow} or \href{redfin.com}{Redfin} --- Two tech companies specializing the US real-estate market. They provide an estimate for the MV based on the location, number of bedrooms and bathrooms, and a couple other factors they feed into their algorithms. Note that these numbers are only estimates of the MV, so we don't stick by them all the time. Sometimes we use 1.3MV instead of just the MV to account for that. And note that other tech companies provide other estimates, but we stick with the biggest two --- Zillow and Redfin.
    

    \item \bb{Condition:}
    The condition of the property is a big factor in the cash offer the owner gets. If the property requires lots of major repairs, then it's mostly overvalued on Zillow and Redfin. And hence, you should expect a lower AP\@. Some of the major things to ask about and look for are: the \bb{roof}, the \bb{flooring}, \bb{foundation}, \bb{plumbing}, \bb{HVAC}, etc. If on the other hand, the owner has completely renovated the property recently and/or did many upgrades/expansions, you should expect their AP to be closer to the MV or above, and it to be undervalued on Zillow and Redfin.

    \item \bb{Residency:}
    Whether the property is vacant or not affects when the buyer can close on the deal, and how the property will be handled after the purchase. If it has been vacant for quite a bit of time, then it probably needs some cosmetic work even if it was in good condition. And if it had a tenant in there for a long time, then you might expect the closing time to be longer to deal with the tenant. Generally, it is one of three things: \bb{Owner occupied}, \bb{Tenants}, \bb{Vacant}. And if family members or friend live there, you should ask if they are on a lease AKA if they pay rent or not. And asking for the type of lease --- annual or monthly --- is a good extra if you can mange it.

    \item \bb{Listing Status:}
    To sell a property, you have to find a buyer. And finding a buyer isn't always easy. So, you might need the help of a \bb{realtor}. Realtors AKA real estate agents are people who specialize in the real estate business and can help connect buyers to sellers. They also have access to the \bb{MLS} which is a place where all realtors from a certain town/state list their properties. And of course, realtors don't work for free. Realtors have fees. So, if a property is listed with a realtor, you might need to skip that one or inform the owner that we don't cover realtor fees. That depends on the camp you are calling, and the AP\@. Most of the time, the property has to be \textbf{NOT} listed with a realtor for the lead to be qualified. And usually, it will be not listed. Since the lists we are calling are compiled from expired listings.

\newpage

    \item \bb{Closing Time:}
    The closing time is the period between the owner accepting an offer, and them signing the official paper work and moving out. These are determined by the conditions of the owner, and how motivated to sell they are. And it could affect the AP sometimes depending on how the market is doing.
    
    \item \bb{Reason for Selling}
    The most common reason people sell their properties is \$\$\$. People buy properties as investments when the market is down, and sell them when the market is high. But sometimes, people can have other reasons to sell. It might a divorce, the loss of a family member, or them just wanting to move out of state or somewhere else. If an owner has a reason to sell other than the cash offer, then it is more likely for him to be actually interested, and hence a lead.

    \item \bb{Basic Info}
    You should always check and confirm with the owner the basic info about the property, and their personal info. You should check that you two are talking about the same property by confirming: \bb{The address}, and \bb{The number of beds \& baths}. And you should confirm their identity --- \bb{Full name}, and \bb{Phone number}. And you should get some contact info --- \bb{Email}, \bb{Other phone numbers}, and \bb{Best callback time}.
    
\end{enumerate}

\subsection{What disqualifies a lead}
    \vspace{1em}
\begin{enumerate}[label= \roman*.]
    \item \bb{Window shopping}:
    The most common thing that disqualifies a lead is the owner not being actually interested in selling and being more of a window shopper. They just want to get an offer to get a free appraisal, for example. Or they are just curious about the process. These owners will ask for an offer right away, or might go through the process sometimes but won't give good reasons or specific answers to most of your questions.

    \item \bb{Random answers}:
    Owners get calls on their properties all the time. And they get calls from other telemarketers in different fields as well. Sometimes, they may get up to 20 marketing calls per day. So, some owners resort into complying to stop the calls. But they give random/fake answers. And you can filter those out sense most of the time the owner will seem incoherent.

    \item \bb{Failure to meet the criteria for a certain camp}
    Some camps have certain specific criteria you have to abide by. Failure to meet those results in an immediate disqualification.

    \item \bb{Failure to ask/obtain some critical info}
    If you didn't confirm the address, missed to ask about the closing time, or anything else like that. That will get you a \bb{Call Back (CB)}. And unfortunately, not all owners pick up and talk to you again. So, it's very likely to end as a disqualified lead.

\end{enumerate}

\newpage

\section{Script}
Refer back to {\color{blue} New comers' script}

\section{Objection handling}
\bowner{\bb{Wrong Number:} “You have the wrong Number”} \\
I apologize, I was trying to get a hold of x at 123 main st… do you by any chance know how I can get in contact with him or her? \\
(IF NOT)
Do you have any properties that you’ll consider selling or know somebody else that does?

\bowner{\bb{Death:} “He/she passed away”} \\
My condolences, my name is (Insert Caller) who may I possibly speak to about this property?

\bowner{\bb{Not Interested:} “I am not interested in selling”} \\
Do you have any other properties or even land that you’ll consider selling or know somebody else that does? 

\bowner{\bb{How did you get my number?:} “How’d you get my number”}
Well, this number was associated with the property in the public records. When we’re interested in a property we go through there to see who is the current owner so we can express our interest.

\bowner{\bb{Public Records:} Which public records?} \\
The Tax Assessment office at the courthouse.

\bowner{\bb{Off list: } Take me off your list} \\
Well there isn’t any list. I was just calling to express my interest.

\bowner{\bb{DNC\@: } I am on the DNC, why are you calling me?} \\
The DNC call list applies to people who are offering services, or telemarketers that want to sell you something. I am just calling on behalf of myself and my group to express our interest in the property.

\bowner{\bb{Lost interest }I had terrible offers from other investors, I lost interest.} \\
Oh, I’m sorry to hear that. I am not affiliated with the people that called you before. I'm calling on behalf of myself and my group. We’d love to present to you the best offer. So how much were you looking to sell for?


\bowner{\bb{Company} What company are you with?} \\
It's called Good Offer, and I work with a group of investors

\bowner{\bb{What’s your offer?} Why are you calling if you don’t have an offer} \\
I was calling to see if there is an interest, if there is an interest I can run my proper numbers, but what more can you tell me about the property? (beds / baths)

\bowner{\bb{Not at this time} I don't plan to sell now} \\
Do you by any chance have a certain timeline? If you do plan to sell in the near future, we can talk about it now. And maybe we can get something out of it.

\section{Lead temp}
\subsection{Hot lead}
Lead with an owner who's very motivated to sell, has an AP, a reason to sell, and wants to close right away or within 30 days.
\subsection{Warm lead}
Lead with an owner who seems interested, but didn't fill in most of the details yet.
\subsection{cold lead}
Lead with an owner who's not super interested, but might consider it if things fell into place.

\section{Soft skills}

\begin{itemize}
    \item Try to mirror the owner's tone of voice. Calm owner = calm you, loud owner = a bit louder you, etc.
    \item Active listening! Don't ask questions the owner already answered while talking to you.
    \item Don't be married to the script! Try to have an actual convo with the owner. Laugh on their jokes, express sorrow when they say something sorrowful, etc. 
    \item Try to always be on top of the convo with your questions. The more time you give to the owner, the more they'll think of something you might not be able to answer.
    \item Try to keep an upbeat tone of voice. 
\end{itemize}

\section{Background info}
\subsection{Key steps in buying a home}
Refer to \href{https://www.zillow.com/home-buying-guide/10-steps-to-buying-a-home/}{this nice detailed article} on Zillow!
\subsection{How do mortgages work}
Refer to \href{https://www.zillow.com/mortgage-learning/what-is-a-mortgage/}{this nice detailed article} on Zillow!
\subsection{What is wholesaling}
Refer to \href{https://www.investopedia.com/ask/answers/100214/what-goal-real-estate-wholesaling.asp}{this nice detailed article} on Investopedia!

\subsection{Jargon}
\jargon{Down payment} \\
The amount of money the owner paid at first before paying the rest as installments
\vspace{1em}

\jargon{Equity} \\
The amount of the house's value the owner paid off at the moment
\vspace{1em}

\jargon{Closing costs} \\
Fees associated with the purchase of a home that are due at the end of the sales transaction. Fees may include the appraisal, the home inspection, a title search, a pest inspection and more. It comes close to an amount that is 2\% to 5\% of the home’s purchase price.
\vspace{1em}

\jargon{Foreclosure} \\
A property repossessed by a bank when the owner fails to make mortgage payments. 
\vspace{1em}

\jargon{Lien} \\
A lien is any legal claim upon a property for a debt or a non-monetary interest in the property. A lien is a security interest that can give a creditor the right to take possession of a property secured by a loan, such as a mortgage, when the borrower defaults on the loan obligations. Most lenders will require title insurance to protect their interests should there be outstanding liens on the property securing their security interest.
\vspace{1em}

\jargon{Capital gains tax} \\
A type of tax you pay on the profit you made from selling a property or any other type of non-inventory asset like stocks and bonds for example.
\vspace{1em}

\jargon{Under contract} \\
A period of time (typically 30 days or more) after a buyer has made an offer on a home and a seller has accepted. During this time, the home is inspected and appraised, and the title is searched for liens, etc.
\vspace{1em}

\jargon{Zoning} \\
A designation, assigned by local government, to a parcel of land that dictates how it can be used. Common designations include residential, commercial, industrial and agricultural.
\vspace{1em}

\jargon{Tax assessment or assessed value} \\
The value assigned to a home by a local government to determine the amount of property taxes a homeowner owes.  The assessment, which is usually made once a year, differs from an appraisal, which estimates the value of a home, based on market conditions when it’s listed for sale. 
\vspace{1em}

\jargon{Deed} \\
A deed is the legal document that establishes ownership of real property, and is also used to transfer the ownership of real property to another person or entity.

\end{document}
