\documentclass[12pt]{exam}
\usepackage{graphicx} % Required for inserting images
\usepackage[margin = 2cm]{geometry}
\usepackage{amsfonts}
\usepackage{amsmath}
\usepackage{amssymb}
\usepackage{amsthm}

\newcommand{\pcomnum}[2]{\(#1 +i #2\)}
\newcommand{\ncomnum}[2]{\(#1 -i #2\)}
\newcommand{\comcir}[2]{\(\left| z - #1 \right| = #2\)}
\newcommand{\zbar}[0]{\Bar{z}}
\newcommand{\modz}[0]{\left|z\right|}
\newcommand{\cint}[1]{\int_C #1 \ dz}
\newcommand{\ccint}[1]{\oint_C #1 \ dz}
\newcommand{\C}[0]{\(C  \) }
\newcommand{\infsum}[1]{\sum_0^\infty\ #1}
\newcommand{\sing}[1]{\left(z #1\right)}
\newcommand{\mac}[0]{Maclaurin series }
\newcommand{\tay}[0]{Taylor series }
\newcommand{\lau}[0]{Laurent series }
\newcommand{\res}[1]{\underset{#1}{Res \ }}

\DeclareMathOperator{\Log}{Log}

\footer{}{\thepage}{}

\begin{document}

\begin{center}
    \bfseries\itshape\Huge
    SHEET 5
\end{center}
\hrule\vspace{0.2em}\hrule
\vspace{1em}

\begin{questions}
\large
\question\ Determine the location \& order of the zeros for the following functions:
\begin{parts}
    \part\ \({\left(z + 16i\right)}^3\)
    \part\ \(\cot z\)
    \part\ \(\cos z\)
    \part\ \(\left(3z^2+1\right)e^{-z}\)
    \part\ \({\left(z^2 - 1\right)}^2 \left(z + 4\right)\)
\end{parts}

\question\ Find all the singularities and the corresponding residues of the following:
\begin{parts}
    \part\ \(\displaystyle\frac{1}{z^2 + 4}\)
    \part\ \(\displaystyle\frac{\cos z }{z^3}\)
    \part\ \(\displaystyle\frac{z^2 + 1}{z^2 - 2}\)
    \part\ \(\tan z\)
    \part\ \(\displaystyle\frac{z^2}{z^4 - 1}\)
    \part\ \(\displaystyle\frac{2z + 1}{z^2 -z -2}\)
\end{parts}

\question\ Show that all singular points of each of the following functions are poles. Also, determine the order of each pole and the residue of the function at the pole.
\begin{parts}
    \part\ \(\tan z\)
    \part\ \(\displaystyle\frac{1-e^{2z}}{z^4}\)
    \part\ \(\displaystyle\frac{e^z}{z^2 + \pi^2}\)
    \part\ \(\displaystyle\frac{1 - \cosh{z}}{z^3}\)
\end{parts}

\question\ Show that
\begin{parts}
    \part\ \(\res{z=-1} \displaystyle\frac{z^{\frac{1}{4}}}{z+1} = \frac{1 + i}{\sqrt{2}}\) \quad\quad \(\modz > 0 \ ,\quad 0 < \arg z < 2\pi\)
    \part\ \(\res{z=i} \displaystyle\frac{\Log z}{{(z^2 + 1)}^2} = \frac{\pi + 2i}{8}\)\
    \part\ \(\res{z=\pi i} \displaystyle\frac{e^z}{\sinh{z}} + \res{z=-\pi i} \ \displaystyle\frac{e^z}{\sinh{z}} = -2 \cos \pi\)
\end{parts}

\question\ Use Cauchy's residue theorem to evaluate the following:
\begin{parts}
    \part\ \(\displaystyle\ccint{\frac{\sin \pi z}{z^4}}\) \newline\newline where \(C\) is the circle \comcir{i}{2}, described in the positive sense.
    \part\ \(\displaystyle\ccint{\frac{1}{{(z-1)}^3 (z+4)}}\) \newline\newline where \C is the circle \(\modz = 3\), described in the positive sense.
    \part\ \(\displaystyle\ccint{\frac{z+10}{6z^2 +5z +1}}\) \newline\newline where \C is the circle:
    \begin{subparts}
        \subpart\ \(\modz = 1\)
        \subpart\ \(\modz = 2\)
        \subpart\ \(\modz = 3\)
    \end{subparts}

    \part\ \(\displaystyle\ccint{e^{-\frac{1}{z}} \sin{\frac{1}{z}}}\) \newline\newline where \C is the circle \(\modz = 1\), described in the positive sense.
    \part\ \(\displaystyle\ccint{\frac{e^{-z}}{{(z-1)}^5}}\) \newline\newline where \C is the circle \(\modz = 2\), described in the positive sense.
\end{parts}
\question\ Find the value of \[\displaystyle\ccint{\frac{3z^2+2}{(z-1)(z^2+9)}}\] taken counterclockwise around the circle:
\begin{parts}
    \part\ \comcir{2}{2}
    \part\ \(\modz = 4\)
\end{parts}
\question\ If \C is the circle \(\modz = 1\) described in the positive sense, show that:
\begin{parts}
    \part\ \(\displaystyle\ccint{\frac{e^{-z}}{z^2}} = -2\pi i\)
    \part\ \(\displaystyle\ccint{ze^{\frac{1}{z}}} = \pi i\)
\end{parts}
\end{questions}
\end{document}
